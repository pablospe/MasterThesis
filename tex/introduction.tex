\chapter{Introduction}
\label{cha:intro}

\vspace*{-2ex}
In robotics, sensor calibration is extremely important since a robot needs its sensors to measure and interact with the environment.
% becomes less useful without~it.
% a proper calibration.
Examples where calibration is necessary are:
\begin{itemize*}
 \item \textbf{Manipulation}: while grasping, or interacting with the environment using its arms, object detection is a common task. An uncalibrated robot will misgrasp the object leading to a failure.

 \item \textbf{Sensor fusion}: suppose that a sensor produces 3D data, and another sensor, color images. It might be desired to fuse both measurements in order to create a color 3D model. A correct fusion cannot be achieved without a calibrated robot.
\end{itemize*}

\noindent
In this thesis calibration of multiple cameras will be studied. The term \textit{recalibration} refers to the existence of an initial calibration, provided by the robot model description (the URDF). Another important assumption is that one camera is already calibrated with respect to the robot. This camera will be the \textit{reference camera}.


\vspace*{-2ex}
\section{Thesis organization} % and methodology}

The background is presented in chapter \ref{cha:background}, and it is divided in two main sections: first, a \textit{theoretical background} will be explained in section~\ref{sec:theoretical_background}; second, a \textit{technical background} in section~\ref{cha:technical_background}, where ROS (Robot Operative System) \cite{ROS}, RViz (ROS visualizer) \cite{RViz}, PR2 (Personal Robot 2, from Willow Garage) \cite{PR2}, KDL (Kinematics and Dynamics Library) \cite{KDL}, Ceres Solver (non-linear least squares minimizer) \cite{ceres}, and other infinity amount of tools will be briefly explained.

Chapter~\ref{cha:multi-view calibration} explains the process and methods developed in this thesis. \textbf{Real data} obtained from PR2 is used to calibrate the \textit{multiple cameras}; experimental results are also presented.

\textit{Implementation details} will be discussed in chapter~\ref{cha:implementation}. Motion initialization is analyzed in chapter~\ref{cha:additional}, where a novel method is presented and directions for further research are provided. Followed by \textit{future work} in chapter~\ref{cha:future}, and \textit{conclusions} in chapter~\ref{cha:conclusions}.

% Another strong division in this thesis is the use of \textit{synthetic data} (chapter~\ref{cha:stereo_recalibration}) vs \textit{real data} (chapter~\ref{cha:multi-view calibration}). Synthetic data has been used in stereo in order to \textit{validate} the method (posteriorly generalized for more cameras), before endeavor a more complicated task with real data.



\section{Goals}

The goals on this work are:
\begin{itemize*}
  \item the creation of a ROS calibration package for multiple cameras, such as RGB cameras, Microsoft Kinect, Prosilicas (high-definition camera), and other cameras mounted in a robot; in our test case, mounted in the PR2 head;

 \item the estimation of the best relative position between the cameras, using 2D measurements extracted from a known beforehand pattern (checkerboard). The relation among cameras is supposed to remain rigid over time. Even if the robot or its joints move, they are all located in the PR2 head which is a rigid entity.
\end{itemize*}




\section{Related Work}

Many people have developed specific techniques for pairwise calibration between
sensors. In \cite{Zhang04extrinsiccalibration}, line constraints are used to calibrate a 2D laser range-finder to a camera. Similar work was done in \cite{Unnikrishnan_fastextrinsic}, but focused on using plane constraints from a 3D laser range-finder system. A more complex case is when an actuated camera system needs to be calibrated (hand-in-eye system) and has been addressed in \cite{Horaud_hand-eyecalibration}. Articles \cite{Dynamic_camera_calibration} and \cite{4587681} were particularly useful to understand fundamental concepts. In addition, the PR2 calibration package is described in \cite{pr2_calibration_paper}.

