\chapter{Lemma demonstrations}
\label{ap:proof}

This Appendix has auxiliary lemmas used in the theorem (\ref{th:norm}) proof. Knowledge of linear algebra will be assumed.

% \cite{Strang:1993}

\begin{lemma}
\label{lemma:symmetric}
\[ \left.
  \begin{array}{l}
  A = Q\,\Lambda\,Q^T \\
  \Lambda \mbox{ diagonal matrix} \\
  Q \mbox{ symmetric matrix}
  \end{array} \right\}
  \Rightarrow
  A \mbox{ and } \Lambda \mbox{ have same eigenvalues.}
\]
\end{lemma}

\begin{proof} See \cite{Strang:1993} for a demonstration.
\end{proof}


\begin{lemma}
\label{lemma:singular_values}
\[ \left.
  \begin{array}{l}
  E =[t]_{\times}R \\
  ||t||=k
  \end{array} \right\}
  \Rightarrow
  E \mbox{ has two singular values equal to $k$ and the last one is zero.}
\]
\end{lemma}

\begin{proof}
From \cite[Result A4.1 (ii)]{HZ2},
\begin{align*}
S=[t]_{\times} \mbox{ is skew-symmetric } & \quad\Rightarrow\quad S = \alpha\,U\,Z\,U^T, \quad \mbox{ with $U$ orthogonal} \\
        &  \quad\Rightarrow\quad  S = U\,Z_{\alpha}\,U^T
\end{align*}
where
\begin{equation}
Z =
\begin{pmatrix}
  0 & 1  & 0 \\
 -1 &  0  & 0 \\
  0 &  0  & 1
\end{pmatrix}, \quad\quad\quad
Z_{\alpha} =
\begin{pmatrix}
  0 & \alpha  & 0 \\
 -\alpha &  0  & 0 \\
  0 &  0  & 1
\end{pmatrix}
\end{equation}
Taking
\begin{align}
S\,S^T & = U\,Z_{\alpha}\,U^T (U\,Z_{\alpha}\,U^T)^T \nonumber \\
       & = U\,Z_{\alpha}\,(U^T U)\,Z_{\alpha}^T\,U^T \nonumber \\
       & = U\,Z_{\alpha}\,Z_{\alpha}^T\,U^T \label{eq:similar_matrix}
\end{align}
From (\ref{eq:similar_matrix}),
\begin{align}
S\,S^T \mbox{ and } Z_{\alpha}\,Z_{\alpha}^T \mbox{ are similar matrices } & \quad\Rightarrow\quad tr(S\,S^T) = tr(Z_{\alpha}\,Z_{\alpha}^T) \label{eq:equal_trace}
\end{align}
By expansion,
\begin{equation}
S\,S^T =
\begin{pmatrix}
  t_3^2 + t_2^2 & 0  & 0 \\
  0 &  t_1^2 + t_3^2  & 0 \\
  0 &  0  & t_2^2 + t_1^2
\end{pmatrix}
\end{equation}
then,
\begin{equation}
\label{eq:S_trace}
tr(S\,S^T) = (t_1^2 + t_2^2 + t_3^2) + (t_1^2 + t_2^2 + t_3^2) = 2\,||t||^2 = 2\,k^2
\end{equation}
In the other hand (by expansion),
\begin{equation}
\label{eq:Z_trace}
tr(Z_{\alpha}\,Z_{\alpha}^T) =
tr\,\begin{pmatrix}
  \alpha^2 & 0  & 0 \\
  0 &  \alpha^2  & 0 \\
  0 &  0  & 0
\end{pmatrix} = 2\,\alpha^2
\end{equation}
From (\ref{eq:similar_matrix}), (\ref{eq:S_trace}) and  (\ref{eq:Z_trace}),
\begin{equation}
\alpha^2 = k^2
\end{equation}
and also
\begin{equation}
\label{eq:ZZT}
Z_{\alpha}\,Z_{\alpha}^T =
\begin{pmatrix}
  k^2 & 0  & 0 \\
  0 &  k^2  & 0 \\
  0 &  0  & 0
\end{pmatrix}
\end{equation}
Now, since $Z_{\alpha}\,Z_{\alpha}^T$ diagonal, and $U$ is orthogonal (symmetric), it is possible to use lemma (\ref{lemma:symmetric}) then,
\begin{equation}
\label{eq:SST_same_ZZT}
\mbox{$S\,S^T$ has the same eigenvalues than $Z_{\alpha}\,Z_{\alpha}^T$}
\end{equation}

\begin{center}
\line(1,0){250}
\end{center}

Now, we start from the hypothesis:
\begin{align}
\label{eq:EET_same_SST}
E=S\,R & \quad\Rightarrow\quad E\,E^T = S\,R\,(S\,R)^T = S\,R\,R^T\,S^T = S\,S^T
\end{align}
From (\ref{eq:SST_same_ZZT}) and (\ref{eq:EET_same_SST}),
\begin{equation}
\label{eq:EET_same_ZZT}
\mbox{$E\,E^T$ has the same eigenvalues than $Z_{\alpha}\,Z_{\alpha}^T$}
\end{equation}
Then,
\begin{equation}
E \mbox{ has two singular values equal to $k$ and the last one is zero.}
\end{equation}
and,
\begin{equation}
E = U\, diag(k,k,0)\,V^T            \quad\quad\mbox{a SVD decomposition}
\end{equation}


\end{proof}

